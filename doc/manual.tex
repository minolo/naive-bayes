\section{Manual}

\subsection{Requisitos}

\begin{itemize}
	\item Linux
	\item Bash
	\item Latex (recomendado TeX Live)
	\item Make
	\item Python 3
	\item matplotlib (librería de python)
\end{itemize}

\subsection{Árbol del proyecto}

\begin{itemize}
	\item \texttt{./}: Raíz del proyecto
	\begin{itemize}
		\item \texttt{./Makefile}: Makefile principal
		\item \texttt{./src}: Código del proyecto
		\item \texttt{./tests}: Ficheros y salidas de tests
		\begin{itemize}
			\item \texttt{./tests/makefile}: Makefile específico
				para los tests. Permite lanzar un tests
				específico
		\end{itemize}
		\item \texttt{./doc}: Ficheros \texttt{.tex} y makefile para
			generar la documentación
		\item \texttt{./res}: Recursos
		\begin{itemize}
			\item \texttt{./res/bib}: Bibliografía, script y
				makefile necesarios para descargarla
				automáticamente
			\item \texttt{./res/dsets}: Base de datos de correos y
				makefile necesario para descargarla
		\end{itemize}
	\end{itemize}
\end{itemize}

\subsection{Generación y ejecución}

Mediante el programa \textit{Make} puede automáticamente descargar los recursos
y bibliografía necesarios, lanzar todos los tests y generar esta documentación.
Para ello únicamente debe ejecutar el comando \texttt{make} o \texttt{make all}
y tener paciencia.

Además del objetivo ``all'', puede indicar a make que únicamente genere alguno
de los siguientes objetivos:

\begin{itemize}
	\item \textbf{all:} Genera la documentación y descarga los recursos
	\item \textbf{doc:} Genera esta documentación. Antes ejecuta todos lost
		tests para obtener los \texttt{.png} de las gráficas
	\item \textbf{tests:} Lanza el \textit{Tester} por cada archivo
		\texttt{.test} que se encuentre dentro de la carpet
		\texttt{./tests}
	\item \textbf{download\_res:} Descarga la base de datos de correos, el
		enunciado del trabajo y los artículos recomendados.
	\item \textbf{clean:} Borra todos los datos autogenerados (excepto las
		salidas de los tests)
	\item \textbf{mrproper:} Borra todos los datos autogenerados incluyendo
		las salidas de los tests
\end{itemize}

Con \texttt{make -j<num hilos>} se pueden ejecutar los objetivos en paralelo,
siempre que las dependencias lo permitan.

\subsubsection{Documentación}

La documentación del proyecto que está leyendo ahora mismo se ha creado con
\LaTeX~y la compilación está automatizada mediante \textit{Make}. Dado que se
incluyen imágenes con las gráficas de los tests realizados, antes de generar
este pdf se lanzan automáticamente todos los tests.

Puede generar la documentación con el objetivo ``doc'': \texttt{make doc}

\footnotetext{Todos los módulos principales cuentan con el parámetro `-h' que
imprime una breve explicación de su utilización.}
\def\helpvfn{\value{footnote}}
\def\helpfn{\footnotemark[\helpvfn]}
\subsubsection{Tester}

Un fichero de test consiste en una primera línea indicando el número de veces
que se va a ejecutar cada configuración y una lista de configuraciones,
consistente en líneas que van a pasarse como parámetros al \textit{Launcher}

Todos los tests pueden ser ejecutados en paralelo por \textit{Make} si se llama
con la opción `-j': \texttt{make -j4 tests}

Puede lanzar manualmente el módulo \textit{Tester} indicándole la ruta a un
archivo de tests\helpfn:

\subsubsection{Launcher}

Separa el conjunto de entrenamiento de la forma en la que le indiquemos, entrena
según los parámetros que le pasemos y lo evalúa para devolvernos una matriz de
confusión.

Por defecto toma el 20\% de la base de datos como conjunto de evaluación y el
80\% como conjunto de entrenamiento, usa la estrategia de tokenización ``basic''
y usa un umbral de 0.1. Estos parámetros se pueden cambiar mediante
parámetros\helpfn. Este script espera encontrar la base de datos de correos en
la ruta \texttt{./res/dsets}.

Ejemplo de uso: \texttt{./launcher.sh -u 0.5}

\subsubsection{Splitter}

Divide la base de datos de mensajes según le especifiquemos. Podemos dividir
especificando los porcentajes del conjunto de evaluación y entrenamiento, o
elegir los enron del 1 al 5 como entrenamiento y el sexto como
evaluación\helpfn. Genera 4 archivos con los nombres que se le especifiquen.

Ejemplo de uso: \texttt{python3 splitter.py -e 0.7 -t 0.2 -d ../res/dsets ev\_ham\_list.txt ev\_spam\_list.txt tr\_ham\_list.txt tr\_spam\_list.txt}

\subsubsection{Trainer}

Este módulo necesita recibir dos listas de rutas a correos de spam y ham que
utilizara como base de entrenamiento. Estas se pueden conseguir mediante el
script \textit{Splitter} o con cualquier otra herramienta (el comando
\textit{find} por ejemplo). Como salida genera un fichero binario que codifica
un diccionario de Python, este archivo únicamente puede leerlo los módulos
\textit{Evaluator} y \textit{Classifier}.

Ejemplo de uso: \texttt{python3 trainer.py -a tr\_ham\_list.txt -s tr\_spam\_list.txt -o training\_data.out -k basic}

\subsubsection{Evaluator}

Dado unos datos de entrenamiento y un conjunto de evaluación, este módulo
clasifica cada uno de los correos del conjunto recibido para generar la matriz
de confusión. Los datos de entrenamiento es el fichero generado por el
\textit{Trainer} y el conjunto de evaluación es una lista de rutas a correos
generada por \textit{Splitter}, o por cualquier otra herramienta.

Ejemplo de uso\helpfn: \texttt{python3 evaluator.py -a ev\_ham\_list.txt -s ev\_spam\_list.txt -o training\_data.out}

\subsubsection{Classifier}

Dado un correo y unos datos de entrenamiento, decide si este es o no spam. Los
datos de entrenamiento son los que genera el \textit{Trainer} y el correo es
la ruta a cualquier archivo de texto.

Ejemplo de uso\helpfn: \texttt{python3 classifier.py -i correo.txt -t training\_data.out -u 0.1}
