\section{Algoritmo}

\subsection{Descripción}

\textbf{Naïve-Bayes} es una aplicación del \textit{teorema de Bayes} asumiendo
que todos los eventos a estudio son independientes entre si (de ahí el
calificativo de \textit{ingenuo}).

El teorema de Bayes vincula la probabilidad de $A$ dado $B$ con la probabilidad
de $B$ dado $A$ mediante la siguiente fórmula:

\begin{equation}
	\label{eq:bayes}
	P(A_i | B) = \frac{P(B | A_i) P(A_i)}{P(B)}
\end{equation}

Un clasificador de \textit{Naïve-Bayes} asume que la presencia o ausencia de
una característica particular no está relacionada con la presencia o ausencia
de cualquier otra característica, dada la clase variable. Una ventaja del
clasificador de Bayes ingenuo es que solo se requiere una pequeña cantidad de
datos de entrenamiento para estimar los parámetros (las medias y las varianzas
de las variables) necesarias para la clasificación.

\subsection[Probabilidad de una palabra]{Probabilidad de aparición de una palabra en una categoría}

Para calcular la probabilidad de que una palabra dada ($w$) pertenezca a una
clase u otra (spam o ham) vamos a utilizar la siguiente fórmula:

\begin{equation}
	\label{eq:prob-palabra}
	P(w_i|C) = \frac{count(w_i, C)}{\displaystyle\sum_{w \in V}{count(w, C)}}
\end{equation}

Donde $V$ es el conjunto de las palabras de todas las categorías.

\subsection{Spamicidad de una palabra}

A la probabilidad de que un correo sea spam, sabiendo que contiene una palabra
específica, se le denomina \textbf{spamicity}~\cite{eberhardt2015bayesian}.
Posteriormente combinaremos estas probabilidades para calcular la probabilidad
de que un correo sea spam. Para calcular la \textit{spamicidad de una palabra}
utilizamos la siguiente fórmula, que es una aplicación directa del teorema de
Bayes (ver~\ref{eq:bayes}):

\begin{equation}
	\label{eq:spamicity}
	P(S | W) = \frac{P(W | S) P(S)}{P(W | S) P(S) + P(W | H) P(H)}
\end{equation}

\textbf{Donde:}
\begin{itemize}
	\item $\bm{P(S | W)}$ es la probabilidad de que un correo sea spam si contiene la palabra $W$
	\item $\bm{P(W | S)}$ es la probabilidad de que la palabra $W$ pertenezca a un correo de spam
	\item $\bm{P(W | H)}$ es la probabilidad de que la palabra $W$ pertenezca a un correo de ham
	\item $\bm{P(S)}$ es la probabilidad de que un mensaje sea spam
	\item $\bm{P(H)}$ es la probabilidad de que un mensaje sea ham
\end{itemize}

\subsection{Combinación de probabilidades}

Bajo la asunción de que todas las palabras del correo son \textit{eventos
independiente}, podemos combinar las spamicidades de cada palabra del mensaje
calculadas con la ecuación~\ref{eq:spamicity} mediante la siguiente
fórmula~\cite{graham2003probability}:

\begin{equation}
	\label{eq:comb-probs}
	P(S|D) = \frac{\displaystyle\prod_{w \in D}{P(S|w)}}{\displaystyle\prod_{w \in D}{P(S|w)} + \displaystyle\prod_{w \in D}{\left(1 - P(S|w)\right)}}
\end{equation}

\subsection{Problema de precisión}

Dado que el número de tokens de un mensaje puede ser arbitrariamente grande, los
productorios de la ecuación~\ref{eq:comb-probs} podrían llegar a desbordar el
tipo de dato que se está utilizando. Para minimizar esta carencia de la
tecnología usamos esta fórmula que deriva de la anterior haciendo uso de
logaritmos:

\begin{gather}
	P(S|D) = \frac{1}{1 + e^\eta}\\\nonumber
	\\
	\eta = \sum_{w \in D}{\left[\ln\left(1 - P(S|w)\right) - \ln(P(S|w))\right]}
\end{gather}

\subsection{Problema de probabilidad cero}

Cuando una palabra aparece solo en una categoría, la probabilidad de que
aparezca en la otra es cero. Para lidiar con este problema utilizamos un
suavizado laplaciano en la ecuación~\ref{eq:prob-palabra}:

\begin{equation}
	\label{eq:prob-palabra-laplace}
	P(w_i|C) = \frac{count(w_i, C) + 1}{\displaystyle\sum_{w \in V}{count(w, C)} + |V|} \cite{hovold2005naive}
\end{equation}
